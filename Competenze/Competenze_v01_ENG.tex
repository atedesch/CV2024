\headright{\faPuzzlePiece\ Skills and Abilities}

\textbf{\textsc{Strategic Management in the Context of Digital Business Transformation}} \\
My bachelor's degree in engineering provided me with a solid foundation in mathematical-analytical skills, including knowledge of probability and statistics. Later, I developed a strong interest in economics, leading me to pursue a Master’s Degree in Business Management. The Corporate Strategy course, in particular, gave me a broad understanding of key corporate strategies, deepening my theoretical and conceptual knowledge to analyze and define decision-making processes. I gained a comprehensive analytical framework that considers portfolio optimization choices for multi-business companies and the associated strategic and financial implications. \\
Observing the evolution of the competitive landscape, I focused on analyzing the digital transformation process of businesses. Through the Digital Business Transformation course, I was able to analyze the stages and strategies driving the digitalization process and its impact on competitiveness, customer relationships, asset management, human resources, and redefining the value proposition. \\
For the final exam, I collaborated with my team to create a strategic planning document for the "After-sales" phase for Alfa Romeo. The project was later presented at Accenture Digital in Milan and received commendation from the committee for the PowerPoint presentation I prepared. The presentation is available at the following link: \\
\href{https://youtu.be/wcvSV6oRpbU}{youtu.be/wcvSV6oRpbU}

\is
\textbf{\textsc{Data Analysis and Model Development}} \\
With both analytical-mathematical tools (bachelor's degree in Engineering) and economic-strategic skills (Master’s in Business Management) at my disposal, choosing the Master’s in Big Data and Management represented the ideal synthesis and recap of both educational paths. \\
Working with Big Data requires highly interdisciplinary skills, such as experience in programming and computer science, knowledge of advanced statistical techniques, a deep understanding of Business Management, and communication skills. \\
The Master’s program equipped me to create analytical models and interpret them with a business-oriented perspective. Specifically, I developed:
\begin{itemize}
    \item Skills to collect, process, and extract value from large and diverse datasets;
    \item The ability to work with various computer tools to address complex problems;
    \item The ability to understand, visualize, and communicate results to top management;
    \item The capability to create data-driven solutions that increase profits, reduce costs, and improve efficiency.
\end{itemize}
These skills were practically applied in a project I presented as the final project of the Master’s program, developed in collaboration with a team for Cassa Depositi e Prestiti. The project aimed to create a predictive model to provide insights into employees' intentions to resign or stay within the company. \\
To achieve this goal, my team and I:
\begin{itemize}
    \item Prepared the dataset provided by CDP: a database of about 1,000 employees and their associated characteristics (education, seniority, city of residence, etc.);
    \item Implemented three different Machine Learning models (Random Forest, Support Vector Machine Classifier, Logistic Regression);
    \item Presented the results, demonstrating a web interface prototype, developed in Flask, D3.js, Python, HTML, and CSS. This application provided the user with useful insights on resignation probability (running the models in the background).
\end{itemize}

\newpage

\headright{\faPuzzlePiece\ Skills and Abilities}
\textbf{\textsc{Technical and IT Skills}} \\
Thanks to my educational background and personal passion, I have developed proficiency in various IT technologies over time. Below is a non-exhaustive list of skills acquired:
\begin{itemize}
    \item Advanced use of online conference software and remote work tools (Cisco Webex, Zoom, Teams, Google Meet, Discord, etc.);
    \item Use of tools for digital work organization; for example, task management software (Trello, Monday, Jira) and instant messaging software (Slack);
    \item Advanced use of cloud systems for backup and cloud work (Google Drive, Microsoft 365, Dropbox, iCloud drive, Zoho docs);
    \item Knowledge and use of programming languages and data analysis software (Python, R, Google Analytics, PySpark, MatLab, etc.);
    \item Use of advanced data visualization tools (Tableau, GGPlot, etc.);
    \item Knowledge and use of machine learning programming languages (TensorFlow);
    \item Knowledge and use of databases (Access, Hadoop, mySQL).
\end{itemize}
The proficiency acquired in IT allows me to quickly and easily learn new software and tools.
During the Master’s program, I specifically studied and deepened my knowledge of data analysis software, presenting several projects for the final exams of individual courses.
During the same period, due to the pandemic, I also made daily use of online conference software, cloud systems, and applications for digital work organization.

\is
\textbf{\textsc{Organizational Management and Project Management}} \\
In the organizational design course, I acquired knowledge on designing and adopting new organizational models in response to the current competitive, social, technological, and economic landscape. I particularly focused on:
\begin{itemize}
    \item Applying organizational models on a global scale (multinational companies, crowd-based organizations, etc.)
    \item Analyzing complex organizational problems and possible solutions for organizational design: new structures were analyzed with a multi-level, non-linear, and longitudinal perspective to highlight possible complexity elements for management.
\end{itemize}
These skills were applied in real company case discussions, through which I developed experience and problem-solving abilities.
To respond to an increasingly uncertain environment, many companies resort to “projectification.” To refine my training, I focused on analyzing so-called “PBOs” (Project-Based Organizations).
The Project Management course enabled me to:
\begin{itemize}
    \item Understand the coordination mechanisms of the involved actors (project manager, project worker, line manager, HR specialists)
    \item Familiarize myself with tools to initiate, plan, execute, monitor, and close projects (GANTT, PERT, WBS, etc.)
\end{itemize}
I applied the analysis of complex organizational problems in my thesis, highlighting how mutual trust among actors and stakeholders is a major challenge in PBOs. I then proposed a solution for an effective traceability and coordination system; this solution is based on the implementation of a blockchain-based platform, which can increase security and trust among all participants thanks to decentralization, process automation, visibility, and immutability, thereby providing a common vision of reality for the stakeholders involved.
