%\begin{tcolorbox}[colframe=blue!70, colback=blue!10, title=Esempio di riquadro]
%\end{tcolorbox}

%\begin{mdframed}[linewidth=1pt, linecolor=cvblue, backgroundcolor=white]
%\headtitle{Competenze e Abilità}
%\end{mdframed}

\headright{\faPuzzlePiece\ Competenze e Abilità}

\textbf{\textsc{Strategic Management nell’ottica della Digital Business Transformation}} \\
La laurea triennale in ingegneria mi ha fornito solide basi matematico-analitiche e conoscenze nel calcolo probabilistico e statistico. In seguito, i miei interessi si sono rivolti all’approfondimento degli studi economici. Ho quindi intrapreso il percorso di Laurea Magistrale in Gestione d’Impresa; il corso di Strategie d’Impresa, in particolare, mi ha fornito conoscenze trasversali in merito alle principali strategie di corporate delle imprese, attraverso l’approfondimento della strumentazione teorica e concettuale per la comprensione e la definizione del processo decisionale. Ho acquisito un framework di analisi completo che tiene in considerazione le scelte di ottimizzazione del portafoglio delle imprese multi-business e i relativi riflessi di ordine strategico e finanziario. \\
Guardando all’evoluzione del contesto competitivo, mi sono focalizzato sull’analisi del processo di trasformazione digitale del business; grazie al corso di Digital Business Transformation, infatti, ho potuto analizzare le fasi e le strategie che guidano il processo di digitalizzazione e il suo impatto sulla dimensione competitiva, sul rapporto con il cliente, sulla gestione degli assets, sulla gestione delle risorse umane e sulla ridefinizione della value proposition. \\
Per l’esame finale, in collaborazione con il mio team, ho realizzato una documentazione di pianificazione strategica della fase di “After-sales” per l’Alfa Romeo. Il progetto è stato poi presentato presso la sede di Accenture Digital a Milano, con un encomio da parte della commissione per la presentazione PowerPoint da me elaborata.
La presentazione è disponibile al seguente link: \\
\href{https://youtu.be/wcvSV6oRpbU}{youtu.be/wcvSV6oRpbU}

\is
\textbf{\textsc{Analisi dei dati e realizzazione di modelli}} \\
Avendo quindi a mia disposizione sia gli strumenti analitico-matematici (laurea triennale in Ingegneria) che economico-strategici (laurea Magistrale in Gestione d’Impresa), la scelta del Master in Big Data e Management ha rappresentato l’ideale sintesi e ricapitolazione dei due percorsi formativi. \\
Lavorare con i Big Data, infatti, richiede una serie di competenze altamente interdisciplinari, quali l’esperienza nella programmazione e nell'informatica, la conoscenza delle tecniche statistiche avanzate, la comprensione approfondita della Gestione d’Impresa e la capacità di comunicazione. \\
Il Master mi ha quindi messo in grado di creare modelli analitici e interpretarli con una prospettiva orientata al business. In particolare, ho acquisito:
\begin{itemize}
	\item Competenze per raccogliere, elaborare ed estrarre valore da grandi e diversi set di dati;
	\item La capacità di lavorare con diversi strumenti informatici per affrontare problemi complessi;
	\item La capacità di comprendere, visualizzare e comunicare i risultati al top management;
	\item La capacità di creare soluzioni guidate dai dati che aumentino i profitti, riducano i costi e migliorino l'efficienza.
\end{itemize}
Le suddette competenze hanno trovato una concreta applicazione in un progetto che ho presentato come prova conclusiva del Master, in collaborazione con un team di colleghi, per Cassa Depositi e Prestiti. Il progetto aveva come scopo la realizzazione di un modello predittivo in grado di fornire indicazioni sull’intenzione dei dipendenti di licenziarsi o restare in azienda.
Per raggiungere tale scopo, con il mio team, abbiamo:
\begin{itemize}
	\item Preparato il set di dati che CDP ci ha fornito: un database di circa 1.000 dipendenti e le caratteristiche a essi associate (formazione, seniority, città di residenza e domicilio, etc.);
	\item Implementato tre diversi modelli di Machine Learning (Random Forest, Support Vector Machine Classifier, Logistic Regression);
	\item Presentato i risultati ottenuti, mostrando un prototipo di interfaccia web, sviluppata in Flask, D3.js, Python, HTML e CSS: questa applicazione forniva all'utente approfondimenti utili sulla probabilità di dimissioni (eseguendo in background i modelli).
\end{itemize}


\newpage

\headright{\faPuzzlePiece\ Competenze e Abilità}
\textbf{\textsc{Abilità tecnico-informatiche}} \\
Grazie ai miei percorsi formativi e alla passione personale, ho perfezionato nel tempo l’utilizzo di svariate tecnologie informatiche. Di seguito, un elenco non esaustivo delle competenze sviluppate: 
\begin{itemize}
	\item Utilizzo avanzato di software per le conferenze online e lo smart working (Cisco Webex, Zoom, Teams, Google Meet, Discord, etc.);
	\item Utilizzo di strumenti per l’organizzazione digitale del lavoro; ad esempio, software per la gestione dei task (Trello, Monday, Jira) e software di messaggistica istantanea (Slack);
	\item Utilizzo avanzato di sistemi in cloud per il backup e il lavoro in cloud (Google Drive, Microsoft 365, Dropbox, iCloud drive, Zoho docs);
	\item Conoscenza e utilizzo di linguaggi di programmazione e software per l’analisi dei dati (Python, R, Google Analytics, PySpark, MatLab, etc.);
	\item Utilizzo di strumenti avanzati per la visualizzazione dei dati (Tableau, GGPlot, etc.);
	\item Conoscenza e utilizzo di linguaggi di programmazione per il Machine Learning (TensorFlow);
	\item Conoscenza e utilizzo di database (Access, Hadoop, mySQL).
\end{itemize}
La dimestichezza acquisita in ambito informatico mi consente di apprendere con prontezza e facilità l’utilizzo di nuovi software e strumenti.
Durante il Master, ho studiato e approfondito in particolare l’uso di software per l’analisi di dati presentando svariati progetti per gli esami conclusivi dei singoli corsi.
Nello stesso periodo, a causa della pandemia, ho inoltre utilizzato quotidianamente software per le conferenze online, sistemi in cloud e applicativi per l’organizzazione digitale del lavoro.

\is
\textbf{\textsc{Organizzazione aziendale e Project Management}} \\
Nell’ambito del corso di progettazione organizzativa ho acquisito conoscenze sulla progettazione e l’adozione di nuovi modelli organizzativi, in risposta all’attuale scenario competitivo, sociale, tecnologico ed economico. Ho approfondito quindi in particolare: 
\begin{itemize}
	\item L’applicazione di modelli organizzativi su scala globale (imprese multinazionali, organizzazioni di tipo crowd-based, etc.)
	\item L’analisi dei problemi organizzativi complessi e delle possibili soluzioni di progettazione organizzativa: le nuove strutture sono state analizzate adottando una prospettiva multi-livello, non-lineare e longitudinale, al fine di sottolineare anche i possibili elementi di complessità che la loro adozione comporta per il management
\end{itemize}
Le suddette competenze sono state applicate nella discussione di casi aziendali reali, grazie ai quali ho sviluppato esperienza e capacità di problem-solving.
Per rispondere a un contesto sempre più incerto, molte aziende ricorrono alla “progettificazione”: quindi, per affinare la mia formazione, mi sono focalizzato sull’analisi delle c.d. “PBOs” (Project-Based Organizations).
Il corso di Project Management mi ha permesso di:
\begin{itemize}
	\item Comprendere le peculiarità dei meccanismi di coordinamento degli attori coinvolti (project manager, project worker, manager di linea, specialisti delle risorse umane)
	\item Avere familiarità con gli strumenti per avviare, pianificare, eseguire, monitorare e chiudere i progetti (GANTT, PERT, WBS, ecc.)
\end{itemize}
Ho messo in pratica l’analisi di problemi organizzativi complessi nella mia tesi di laurea evidenziando come il problema della mutua fiducia tra gli attori e gli stakeholder coinvolti sia una delle maggiori criticità delle PBOs. Ho quindi proposto una soluzione per un sistema efficace di tracciabilità e di coordinamento; tale soluzione si fonda sull’implementazione di un progetto di una piattaforma basata sulla c.d. Blockchain, la quale può aumentare la sicurezza e la fiducia tra tutti i partecipanti grazie alla decentralizzazione, l’automazione dei processi, la visibilità, l’immutabilità, garantendo di conseguenza una visione comune della realtà per gli stakeholders coinvolti.