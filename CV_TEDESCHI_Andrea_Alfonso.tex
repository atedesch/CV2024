%%% INFO %%%
% CV di Andrea Alfonso Tedeschi
% Aggiornato al 2024-11-26
% Versione 17.3

\documentclass[11pt, a4paper]{article} 

%%% LINGUA DEL CURRICULUM %%%
\newcommand{\lingua}{ITA} % Cambia la lingua
% \newcommand{\versione}{v01} %cambia la versione
% \newcommand{\colonna}{Colonna_destra} % Cambia la colonna

%%% PACKAGE UTILIZZATI %%%
\usepackage[T1]{fontenc}     % We are using pdfLaTeX,
\usepackage[utf8]{inputenc}  % hence this preparation
\usepackage[italian]{babel}  
\usepackage[left = 0mm, right = 0mm, top = 0mm, bottom = 0mm]{geometry}
\usepackage[stretch = 25, shrink = 25, tracking=true, letterspace=30]{microtype}  
\usepackage{graphicx}        % To insert pictures
\usepackage{xcolor}          % To add colour to the document
\usepackage{marvosym}        % Provides icons for the contact details
\usepackage{fontawesome5}    % Pacchetto di icone aggiuntive
\usepackage{tikz}            % Pacchetto 1 per le ombreggiature
\usetikzlibrary{shadows}     % Pacchetto 2 per le ombreggiature
% alcuni stili... diciamo predefiniti: si differenziano per il tipo di ombreggiatura
\tikzset{riquadro 1/.style={inner sep=2pt,draw, fill=white,drop shadow}}
\tikzset{riquadro 2/.style={inner sep=2pt,draw, fill=cvblue,drop shadow}}
% comando per impostare il tipo di riquadro
\newcommand{\usariquadro}[1]{\tikzset{riquadro/.style={#1}}}
% ad esempio
\usariquadro{riquadro 1}

% comando per inserire la figura con riquadro ed ombreggiatura
\newcommand{\riquadraombreggia}[2][]{
\tikz[baseline=-0.5ex]\node[riquadro]{\includegraphics[#1]{#2}};
}

\newcommand{\frameshadow}[2][]{
\tikz[baseline=0ex]\node[riquadro 2]{\includegraphics[#1]{#2}};
}




\usepackage{enumitem}        % To redefine spacing in lists
\setlist{parsep = 0pt, topsep = 0pt, partopsep = 1pt, itemsep = 1pt, leftmargin = 6mm}

\usepackage{FiraSans}        % Change this to use any font, but keep it simple
\renewcommand{\familydefault}{\sfdefault}

\usepackage{tcolorbox}       % Per creare riquadri colorati
\usepackage{mdframed}        % Per creare riquadri colorati

\usepackage{xcolor}			% Per impostare i colori

\usepackage[colorlinks = true, urlcolor = white, linkcolor = white]{hyperref} % Per gestire i link ipertestuali
\definecolor{cvblue}{HTML}{304263}

\usepackage{fp}              % Per fare calcoli numerici

\usepackage{ifthen}

%%% USER COMMAND DEFINITIONS %%%
\newcommand{\dates}[1]{\hfill\mbox{\textbf{#1}}} % Bold stuff that doesn’t got broken into lines
\newcommand{\is}{\par\vskip.5ex plus .4ex} % Item spacing
\newcommand{\smaller}[1]{{\small$\diamond$\ #1}}
\newcommand{\headleft}[1]{\vspace*{3ex}\textsc{\textbf{#1}}\par%
    \vspace*{-1.5ex}\hrulefill\par\vspace*{0.7ex}}
\newcommand{\headright}[1]{\vspace*{2.5ex}\textsc{\Large\color{cvblue}#1}\par%
     \vspace*{-2ex}{\color{cvblue}\hrulefill}\par}
\newcommand{\headtitle}[1]{\vspace*{2.5ex}\textsc{\Large\color{white}#1}\par%
     \vspace*{-2ex}{\color{red}\hrulefill}\par}

%%% INIZIO DOCUMENTO %%%
\begin{document}

% Style definitions -- killing the unnecessary space and adding the skips explicitly
\setlength{\topskip}{0pt}
\setlength{\parindent}{0pt}
\setlength{\parskip}{0pt}
\setlength{\fboxsep}{0pt}
\pagestyle{empty}
\raggedbottom

\begin{minipage}[t]{0.33\textwidth} % Left column -- outer definition
%------------------------------------------------------------------  Left column -- top dark rectangle
\colorbox{cvblue}{\begin{minipage}[t][5mm][t]{\textwidth}\null\hfill\null\end{minipage}}

\vspace{-.2ex} 						% Eliminates the small gap
\colorbox{cvblue!90}{\color{white}  % LEFT BOX
	\kern0.09\textwidth\relax		% Left margin provided explicitly
	\begin{minipage}[t][293mm][t]{0.82\textwidth}
	\raggedright
	\vspace*{2.5ex}

%------------------------------------------------------------------ Intestazione e foto
	\Large Andrea Alfonso \textbf{\textsc{Tedeschi}} \normalsize 

	\begin{center}
	\riquadraombreggia[width=0.8\textwidth]{SX/Foto_profilo.jpg}
	\end{center}

	\vspace*{0.5ex} % Extra space after the picture

%------------------------------------------------------------------ Contatti
	\headleft{Contatti}
	\small % To fit more content
	\MVAt\ {\small andrea.a.tedeschi@gmail.com} \\[0.4ex]
	\Telefon\ +39\,331\,457\,3469 \\[0.5ex]
	\faLinkedin\ \href{https://www.linkedin.com/in/aatedeschi/}{linkedin.com/in/aatedeschi/} \\[0.1ex]
	\faInstagram\ \href{https://www.instagram.com/andrea.teddy8/}{instagram.com/andrea.teddy8/} \\[0.1ex]

%------------------------------------------------------------------ In breve
	\headleft{In Breve}
Professionista con oltre 10 anni di esperienza nella consulenza strategica e trasformazione digitale, specializzato in analisi dei dati e sviluppo di soluzioni innovative per migliorare efficienza e competitività aziendale. Solide competenze in machine learning, gestione del rischio e modelli di business, maturate grazie a percorsi formativi avanzati e progetti complessi per aziende e pubblica amministrazione. Orientato al problem-solving, con un approccio analitico e una visione strategica per sostenere start-up e grandi realtà nel percorso di digitalizzazione.

%------------------------------------------------------------------ Conoscenze Linguistiche
	\headleft{\faLanguage\ Conoscenze linguistiche}
	\begin{itemize}
		\item \textbf{Italiano}~-- madrelingua \\
		\item \textbf{Inglese}~-- avanzato \\
		\item \textbf{Spagnolo}~-- principiante \\
	\end{itemize} 

%------------------------------------------------------------------ Competenze informatiche
	\headleft{\faLaptop\ Competenze informatiche}
	Windows, Mac, Linux, Word, Excel, Powerpoint, Access, 
	Python, C, R, Matlab, MySQL, Hadoop, PySpark, LaTeX, WordPress, 
	PowerBi, Tableau.

%------------------------------------------------------------------ Certificazioni
	\headleft{\faCertificate\ Certificazioni}
	\begin{itemize}
		\item Public Speaking
		\item Cisco Academy
		\item Skill-a-bus
		\item Bloomberg Market Concepts
	\end{itemize}

	\end{minipage}%
	\kern0.09\textwidth\relax
} 									% End of the colorbox
\end{minipage} 						% Left column -- End of the outer definition
\hskip2.0em						% Left margin for the white area
\begin{minipage}[t]{0.58\textwidth}
\setlength{\parskip}{0.8ex}% Adds spaces between paragraphs; use \\ to add new lines without this space. Shrink this amount to fit more data vertically

\vspace{2ex}

%------------------------------------------------------------------ Esperienze Professionali
\headright{\faUserTie\ Esperienze Professionali}

\textsc{\textbf{Senior Consultant}} \dates{02/2023 -- presente} \\ 
\textbf{\textit{Luiss Business School}} | Roma \\
Università privata di economia e management \\
\smaller {Consulente in progetti di Digital Transformation in organizzazioni complesse.} \\
\smaller{Reingegnerizzazione dei processi; definizione della catena del valore; sviluppo degli strumenti di monitoraggio; mappatura dei servizi in prospettiva AS IS e TO BE; digitalizzazione, analisi e disegno dei processi e dei modelli operativi a supporto a Enti e realtà della Pubblica Amministrazione su progettazione.}


\is
\textsc{\textbf{Consulente strategico e Data analyst}} \dates{09/2021 -- 02/2023} \\ 
\textbf{\textit{Lux Made In}} | Roma  \\
Agenzia di consulenza per start-up \\
\smaller{Consulente per start-up innovative 
per permettere loro di accedere ai fondi messi a disposizione da Invitalia.}


\is
\textsc{\textbf{Consulente strategico e Data analyst}} \dates{04/2020 -- 09/2021} \\ 
\textbf{\textit{Estate2Rent}} | Milano  \\
Piattaforma innovativa di locazione a lungo termine \\
\smaller{Supporto al management nel processo di avviamento.} \\
\smaller{Definizione del modello di business, dell'assetto organizzativo e del framework di processi al supporto del business} \\
\smaller{Analista dati e sviluppatore di “web scraping tool” per ottimizzare 
il prezzo dell’affitto in relazione alla domanda di appartamenti nella città di Milano.}


\is
\textsc{\textbf{Fondatore e Socio}} \dates{12/2012 -- 09/2017} \\ 
\textbf{\textit{Istituto Nazionale di Cultura}} | Milano \\
Promozione eventi culturali e artistici \\
\smaller{Fondatore della società, definizione della mission, 
sviluppo di business plan, stesura bilancio societario.}


%\is
%\textsc{Research support} at \textit{Everlasting Gobstopper Ltd.\ (Mongolia).} \\ 
% \null is necessary here because this is a manually enforced break
% and \dates start with an \hfill that needs a \nukk
%\null\dates{2016.05--2018.01} \\[-\baselineskip]
%\smaller{Developing and implementing methods for inferring \\
%causal networks from time-series, analysis of customer feedback data, mathematical optimisation, machine learning.}

%------------------------------------------------------------------ Formazione e Istruzione
\headright{\faUniversity\ Formazione e Istruzione}

\textsc{\textbf{Master II livello in Banking and Risk Management}} \\ 
\textbf{\textit{UniCusano}} %\dates{12/2023 -- 09/2017} 
\\
\smaller{Analisi del rischio di credito, Analisi di derivati, Banche e intermediari finanziari.} \\
\smaller{Tesi: I derivati di Criptovalute.}

\is
\textsc{\textbf{Master I livello in Big Data and Management}} \\ 
\textbf{\textit{LUISS Business School}} | Roma  %\dates{12/2023 -- 09/2017}
\\
\smaller{Machine Learning, Big Data Programming Models, Statistics, Python for Finance.}

\is
\textsc{\textbf{Corso intensivo in Fusioni e Acquisizioni}} \\ 
\textbf{\textit{HEC}} | Parigi  %\dates{12/2023 -- 09/2017}

\is
\textsc{\textbf{Laurea Magistrale in Gestione d'Impresa (LM-77)}} \\ 
\textbf{\textit{LUISS}} | Roma  %\dates{12/2023 -- 09/2017}
\\
\smaller{Tesi: La Blockchain nel Project Management e la nuova architettura di fiducia.}

\is
\textsc{\textbf{Laurea Triennale in Ingegneria dei Trasporti (L-7)}} \\ 
\textbf{\textit{La Sapienza}} | Roma  %\dates{09/2008 -- 12/2012}
\\
\smaller{Tesi: La modellistica nell'armamento ferroviario nello studio del comportamento dinamico del binario.}
%\smaller{Lorem ipsum.}

\end{minipage}

\newpage
\newgeometry{left=20mm, right=20mm, top=18mm, bottom=20mm}  % Cambia i margini da ora in poi
\hypersetup{
    linkcolor=cvblue,        % Colore dei link interni
    urlcolor=cvblue         % Colore dei link esterni
}
\setlength{\parskip}{0.8ex}
\FPeval\mynum{0.98}

%%%PAGE 1%%%
\headright{\faTheaterMasks\ \ifthenelse{\equal{\lingua}{ITA}}{Attività Televisiva e Teatrale}{Television and Theatrical Activity}}
\textbf{\textsc{Concorrente}} \\
\textbf{Avanti un Altro} | 24 Febbraio 2022 \\
Quiz a premi in onda su Canale 5, condotto da Paolo Bonolis.
\begin{center}
	\frameshadow[width=\mynum\textwidth]{TV/04_avanti.jpg}
    \label{fig:avanti_un_altro}	
\end{center}
\vspace{2ex}
\textbf{\textsc{Contestant}} \\
\textbf{L'Eredità} | December 26, 2020 \\
A prize quiz show aired on Rai 1, hosted by Flavio Insinna.
\begin{center}
	\frameshadow[width=\mynum\textwidth]{04_TV/eredita.jpg}
    \label{fig:eredita}	
\end{center}

%%%PAGE 2%%%
\newpage
\headright{\faTheaterMasks\ \ifthenelse{\equal{\lingua}{ITA}}{Attività Televisiva e Teatrale}{Television and Theatrical Activity}}
\textbf{\textsc{Performer}} \\
\textbf{Viva RaiPlay!} | 22 Novembre 2019 \\
Varietà in onda su RaiPlay condotto da Rosario Fiorello. \\
Link al video: \href{https://www.dailymotion.com/video/x7qsf6z}{dailymotion.com/video/x7qsf6z}
\begin{center}
	\frameshadow[width=\mynum\textwidth]{TV/02_vivaraiplay.jpg}
    \label{fig:vivaraiplay}	
\end{center}

\vspace{2ex}
\textbf{\textsc{Performer e Regista}} \\
\textbf{Mamma Mia!} | 26 Giugno 2012 \\
Musical a teatro.
\begin{center}
	\frameshadow[width=\mynum\textwidth]{TV/01_mamma.jpg}
    \label{fig:mamma_mia}	
\end{center}

\newpage
\newgeometry{left=20mm, right=20mm, top=18mm, bottom=20mm}  % Cambia i margini da ora in poi
\hypersetup{
    linkcolor=cvblue,        % Colore dei link interni
    urlcolor=cvblue         % Colore dei link esterni
}
\setlength{\parskip}{0.8ex}

\input{Competenze/Competenze_v01_\lingua.tex}

%\newpage
\newgeometry{left=20mm, right=20mm, top=18mm, bottom=20mm}  % Cambia i margini da ora in poi
\hypersetup{
    linkcolor=cvblue,        % Colore dei link interni
    urlcolor=cvblue         % Colore dei link esterni
}
\setlength{\parskip}{0.8ex}
\FPeval\mynum{0.98}

%%%PAGE 1%%%
\headright{\faTheaterMasks\ \ifthenelse{\equal{\lingua}{ITA}}{Attività Televisiva e Teatrale}{Television and Theatrical Activity}}
\textbf{\textsc{Concorrente}} \\
\textbf{Avanti un Altro} | 24 Febbraio 2022 \\
Quiz a premi in onda su Canale 5, condotto da Paolo Bonolis.
\begin{center}
	\frameshadow[width=\mynum\textwidth]{TV/04_avanti.jpg}
    \label{fig:avanti_un_altro}	
\end{center}
\vspace{2ex}
\textbf{\textsc{Contestant}} \\
\textbf{L'Eredità} | December 26, 2020 \\
A prize quiz show aired on Rai 1, hosted by Flavio Insinna.
\begin{center}
	\frameshadow[width=\mynum\textwidth]{04_TV/eredita.jpg}
    \label{fig:eredita}	
\end{center}

%%%PAGE 2%%%
\newpage
\headright{\faTheaterMasks\ \ifthenelse{\equal{\lingua}{ITA}}{Attività Televisiva e Teatrale}{Television and Theatrical Activity}}
\textbf{\textsc{Performer}} \\
\textbf{Viva RaiPlay!} | 22 Novembre 2019 \\
Varietà in onda su RaiPlay condotto da Rosario Fiorello. \\
Link al video: \href{https://www.dailymotion.com/video/x7qsf6z}{dailymotion.com/video/x7qsf6z}
\begin{center}
	\frameshadow[width=\mynum\textwidth]{TV/02_vivaraiplay.jpg}
    \label{fig:vivaraiplay}	
\end{center}

\vspace{2ex}
\textbf{\textsc{Performer e Regista}} \\
\textbf{Mamma Mia!} | 26 Giugno 2012 \\
Musical a teatro.
\begin{center}
	\frameshadow[width=\mynum\textwidth]{TV/01_mamma.jpg}
    \label{fig:mamma_mia}	
\end{center}

\end{document}